\section{Cluster-First Route-Second Method}

 Este método faz uma simples clusterização do conjunto de vértices e então determina uma rota de
veículos em cada cluster. Descreveremos o próximo algoritmos:

\begin{itemize}
\item Fisher and Kaikumar
\item The Petal algorithm
\item The Sweep algorithm
\item Taillard
\end{itemize}

\subsection{Algoritmo de Fisher and Jalkumar}

 O algoritmo de Fisher e Jaikumar {\color{red} referência} é bem conhecido. Resolve um Problema
Genérico de Tarefa (GAP) para formar clusters. O número de veículos $k$ é fixo. O algoritmo pode ser
descrito como segue:

\begin{enumerate}
\item Seed Selection. Escolha pontos semente $J_k$ em $V$ para inicializar cada cluster $k$.

\item Allocation of Customers to Seeds. Calcule o custo $d_{ik}$ da alocar cada cliente $i$ para
cada cliente $k$ com
 \[d_{ijk} = \min (c_{0i}+c_{ijk}+c_{J_k0}, c_{0J_k}+c_{J_ki}+c_{i0})-c_{0J_k}+c_{J_k0}.\]

\item Generalized Assignment. Resolver um GAP com custo $d_{ij}$, peso do cliente $q_i$ e capacidade
do veículo $Q$.

\item TSP Solution. Resolve o TSP para cada cluster correspondente para a solução do GAP.

\end{enumerate}

\subsection{Algoritmo Pétala}

 Uma extensão natural do algoritmo de varredura é gerar muitas rotas, chamadas pétalas {\color{red}
referência}, e fazer uma seleção final resolvendo um problema de partição de conjunto da forma:

 \[\min \sum_{k\in S} d_k x_k\]

sujeito a:
 \[\sum_{k\in S} a_{ik} x_k = 1 \quad (i=1, \ldots, n)\]

com 
 \[x_k = {0,1}, k\in S,\]
 onde $S$ é o conjunto de rotas, $x_k=1$ se, e somente se, a rota $k$ pertence à solução, $a_{ik}$ é
o parâmetro binário iqual a 1 somente se o vértice i pertence a rota $k$, e $d_k$ é o custo da rota
da pétala $k$. Se as rotas corresponde a setores contínuos de vértices, então esse problema possui o
propriedade de coluna circular e é resolvido em tempo polinomial {\color{red} referência}.

\subsection{O Algoritmo de varredura}

 O Algoritmo de varredura aplica-se a instâncias planares do VRP. Consiste de duas partes:

\begin{itemize}
\item Split: Clusters possiveis são iniciados formando rotação e centro de raio no deposito
\item TSP: Uma rota de veículo é então obtida para cada cluster resolvendo o TSP.
\end{itemize}

 Algumas implementações incluem uma fase de otimização posterior na qual os vértices são mudados
entre clusters adjacêntes, e as rotas são reotimizadas. Uma implementação simples deste método
é como segue. onde assumimos que cada vértice $i$ é representado por suas coordenadas polares
$(\theta_i, \rho_i)$, onde $\theta_i$ é o ângulo e $\rho_i$ é o comprimento de raio. Atribuindo um
valor $\theta^* = 0$ para um vértice arbitrário $i^*$ e calculando os ângulos remanecentes de
IMG6$(0,i^*)$. Graduando os vértices em ordem crescente de sua IMG2.

\begin{enumerate}
\item Route Initialization. Escolha um não utilizado veículo $k$.
\item Route Construction. Inicie dos vértices não roteados tendo ângulo menor, atribuindo vertices
ao vértice $k$ enquanto sua capacidade ou o comprimento máximo da rota não é excedido.
\item Route Optimization. Otimizar cada rota de veículos separadamente resolvendo o correspondete
TSP (exatamente ou aproximadamente).
\end{enumerate}

\subsection{Algoritmo de Talliard}

 O algoritmo de Talliard {\color{red} referência} define-se vizinhança usando o
$\lambda$-intermudança mecanismo de Geração {\color{red} referência}. Rotas individuais são
reotimizadas usando algoritmo de otimização de {\color{red} referência}. Um traço nobel do algoritmo
de Talliard é a decomposição dos problemas essenciais em subproblemas.

 Em problemas planares, estes subproblemas são obtidos inicialmente particionando os vértices em
setores centradas no deposito, e em regiões concentricas dentro de cada setor. Cada subproblema pode
ser resolvido independentemente, mas movimentos periódicos de vértices para setores adjacêntes são
necessários. Isso faz sentido quando o deposito é centrado e os vértices são uniformemente
distribuidos no plano.

 Para problemas não planares, e para problemas planares não possuindo esta propriedade, o autor
sugere uma método de partição diferênte baseado no calculo da mais curta espação de arborecência
enraizadas no depósito. Esse método de decomposição é particularmente bem conviniente para
implementeções paralelas com subproblemas podem então ser distribuidos através de varios
processadores.

 A combinação dessas estratégias produz excelentes resultados computacionais.

\section{Métode de roteamento seguido de cluster}

 O métode de roteamento seguido de cluster constroi numa primeira fase uma grande rota de TSP,
observando as restrições laterais, e depositando esta rota dentro das possiveis rotas de veículos
numa segunda fase. Essa ideia aplicasse a problemas com uma número livre de veículso. Foi primeiro
posto por Beasley o qual observou que o problema da segunda fase é um problema de caminho mínimo
padrão em grafo acíclico e pode ser assim resolvido em tempo $O(n^2)$. No algoritmo de caminho
mínimo, o custo $d_{ij}$ da viagem entro os nós $i$ e $j$ é igual a $c_{0i}+c_{0j}+l_{ij}$, onde
$l_{ij}$ é o custo da viagem de $i$ a $j$ na rota TSP.

 Não estamos conscientes de algumas experiências computacionais mostrando que a heuristica de
roteamento primeiro seguido de cluster são competitivas com outras aproximações.



