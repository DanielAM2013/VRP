
\chapter{Definição do Problema}


\section{Problema do Caixeiro Viajante (TSP)}

	Um problema computacional bastante conhecido é o do caixeiro viajante: dado um conjunto de
cidades, que devem ser percorridas por um caixeiro, qual a rota que minimiza a distância a
percorrer?
	Um modo direto de resolução deste problema é analisar todas as possibilidades possiveis e
encontrar a com menor distância. Esta abordagem não é a mais aconselhavel uma vez que o número de
operações requeridas para tal cresce exponencialmente com o número de cidades a percorrer.

	Por este motivo se desenvolveu ao longo de varios anos métodos menos custosos para obter ou pelo
menos se aproximar da solução. Listaremos agora as estratégias mais comuns utilizadas

\begin{enumerate}
\item Vizinho mais próximo
\item Dois ótimo
\end{enumerate}


\section{Problema de roteamento de veículos}

	Uma extensão clara do problema do caixeiro viajante é o problema de roteamento de veículos: dado
um conjunto de entregas para clientes, que podem ser atendidos por mais que um veículo, qual o
melhor conjunto de rotas que miniza a distância total percorrida?

	A seguir especificaremos o problema matematicamente.

	Dado $C$ ( conjunto de clientes) tal que $C \subset \mathbb{R}^2$. O algoritmo $Alg(k,C)$
retorna o conjunto $f_i \in \mathbb{N}_i\times C$ de funções ( rotas) tal que $\sum_{i=1}^k\#
\mathbb{N}_i = \#C$ e $\cap_{i=1}^k \mathbb{N}_i= \emptyset$

	Existem algumas variações do problema de roteamento de veículos que serão listadas abaixo
\begin{enumerate}
\item Com carga (VRPC)
\item Com limite de tempo (VRPTW)
\item Com limite de veículos
\end{enumerate}


	Nosso problema é basicamente um VRP que busca minimizar o número de veículos utilizados e a
distância percorrida.



	Dado um conjunto de pontos em um plano devesse determinar a ordem a percorrer estes pontos de
modo a minimizar o custo do percurso.



\chapter{Algoritmos de Poupança}



\chapter{Algoritmos de dois passos}


\chapter{Algoritmos auxiliares}


\section{Espansão de arvoré mínima}
\subsection{Kruscal}
\subsection{Prim}





