\documentclass[a4paper, 12pt]{article}

\usepackage[portuguese]{babel}
\usepackage{preamble}
\usepackage{pgfplots}
\usepackage{verbatim}
\usepackage{algorithm}
\usepackage[noend]{algpseudocode}
\usepackage{caption}
\usepackage{subcaption}

%\pgfplotsset{width=8cm, compat=1.9}

\begin{document}

\thispagestyle{empty}

{\centering
{\sc 
\textbf{
Universidade Federal do Amazonas \\
Faculdade de Tecnologia\\
Projeto Promobile\\[4cm]
Daniel Antela Torquato\\[4cm]
Gerenciamento de rotas para multiplos veículos}}
\vfill
{\bf Manaus-AM\\2015\\}
}

\newpage


\section{Objetivos}

	Este projeto visa a criação e emprego de algoritmos capazes de coordenas de
forma eficiênte um conjunto de VANT's.

	Os principais problemas existentes são
\begin{description}
	\item[TSP] Problema do caixeiro viajante:

	Dado um conjunto de pontos pre-estabelecidos determinar a melhor rota que,
partindo de um destes, percorra todos os pontos apenas uma vez e volte a origem.

	\item[VRP] Problema de roteamento de veículos:

	Dado um conjunto de pontos pre-estabelecidos determinar o melhor conjunto de
rotas que, partindo de um mesmo ponto, percorram todos os pontos apenas uma vez
por qualquer veículo e volte a origem.
\end{description}

	A dificuldade para a resolução dos problemas acima não esta em descobrir um
método de resolução para uma determinada instância ( conjunto de pontos ), mas
em descobrir o procedimento que, em tempo hábil, determine uma solução
satisfatória.

\section{Desenvolvimento}

	Para a avalização dos resultados obtidos utilizaremos um conjunto de pontos
mapeados nas figuras \ref{test1}, \ref{test2} e \ref{test3}, mostradas abaixo.

\begin{figure}[!ht]
\centering
\foreach \x in {1,2,3}{ 
\begin{subfigure}{0.48\textwidth}
\centering
\caption{Beachmark \x}
\begin{tikzpicture}
\begin{axis}[
	width=7cm 
]
	\addplot+[only marks,blue] table {arg/points\x.dat};
	\addplot+[only marks] coordinates {
		(5.06721,3.94673) [2]
	};
\end{axis}
\end{tikzpicture}
\label{test\x}
\end{subfigure}
}
\end{figure}

\newpage

	Nas seções seguintes serão apresentadas as soluções gráficas bem como o
algoritmo e o código realizados para obtenção das mesmas.
	A solução apresentada não é a melhor existente porem foi obtida em uma
fração do tempo necessário para a obtenção da \emph{solução ótima}.


\subsection{Vizinho mais próximo}

	Uma das heurísticas mais clássicas para obter-se uma aproximação para o TSP
consiste no algoritmo do vizinho mais próximo. Inicialmente o conjunto é
delocado ao conjunto $R$ a origem dos pontos, em seguida encontra-se, em $A$ o
ponto mais próximo a $R$, que contêm apenas a origem, e aloca-se este no ponto
de $R$ mais próximo a $A$. Este passo é repetido até que $A$ esteja vazio. O
algoritmo esta descrito abaixo.

\begin{algorithm}[H]
\caption{VIZINHO\_MAIS\_PROXIMO($A$)}
\begin{algorithmic}
 \State $R \leftarrow A_0$
 \State $A \leftarrow A-{A_0}$
 \While{ Se A é não vazio }
  \State $ [p,q] \leftarrow$ PONTO\_PROXIMO($R$,$A$)
  \State DESLOCAR($p$, $R$, $q$)
 \EndWhile
\end{algorithmic}
\end{algorithm}
	onde PONTO\_PROXIMO retorna o par de pontos mais próximo, um pertencendo a
$A$ e outro pertencendo a $R$, e DESLOCAR aloca espaço para o ponto $p$ em $B$
na posição seguinte a $q$.

	Para descobrir os pontos mais próximos entre $R$ e $A$ deve-se utiliza o
seguinte algoritmo

\begin{algorithm}[H]
\caption{PONTO\_PROXIMO($R$,$A$)}
\begin{algorithmic}
	\For{$i$ em $A$}
		\For{$j$ em $B$}
			\State $d_2 \leftarrow $ distância entre $A_i$ e $B_j$
			\If{$d_2 < d_1$}
				\State $d_1 \leftarrow d_2$
				\State $\textrm{indice}_1 \leftarrow i$
				\State $\textrm{indice}_2 \leftarrow j$
			\EndIf
		\EndFor
	\EndFor
\end{algorithmic}
\end{algorithm}

	O algoritmo acima tem um custo computacional de $n\times m$, onde $n$ é o
número de termos em $A$ e $m$ é o número de termos em $B$. No algoritmo inicial
teremos que utilizar o algoritmo o algoritmo de \emph{pontos mais próximos} até
que o conjunto $A$ seja vazio. A cada passo pelo menos um ponto será retirado de
$A$ e adicionado a $B$ o que significa que termos $n$ passos. Além disso no
passo inicial $B$ tem apenas um ponto enquanto $A$ tem $n-1$. No $k$-esimo passo
$B$ terá $k$ pontos enquanto $A$ terá $n-k$ pontos, logo o algoritmo de
\emph{pontos mais próximos} terá um custo de $k\times(n-k)$ operações, logo o
algoritmo completo executará $\sum_{k=1}^{n} k\times (n-k)$ operações que é
igual à $\frac{n\times(n+1)\times(2n+1)}{3}$ dizemos neste caso que o algoritmo
é de ordem $O(n^3)$ pois seu fator dominante para grandes números será $n^3$


\begin{figure}[!ht]
\centering
\foreach \x in {1,2,3}{ 
\begin{subfigure}{\textwidth}
\centering
\begin{tikzpicture}
\begin{axis} [ width=8cm]
\addplot[mark size=0.1pt, blue] table {arg/nearest_neightbor\x.dat};
\addplot+[only marks] coordinates {
	(5.06721,3.94673) [2]
};
\end{axis}
\end{tikzpicture}
\caption{Pontos roteados pelo vizinho mais próximo (\x)}
\label{roteado\x}
\end{subfigure}
}
\end{figure}

\newpage

	A seguir é apresentado a parte principal do algoritmo do vizinho mais
próximo

{\small
\begin{verbatim}
double nearest_neighbor2(route *R) {                                                                                                                            
 double d[3];                                                                                                                                                   
 d[2]=0;                                                                                                                                                        
                                                                                                                                                                
 route::iterator closer[2];                                                                                                                                     
                                                                                                                                                                
 for( route::iterator last=R->begin(); last+1!=R->end(); last++) {                                                                                              
  d[1]=1e+30;                                                                                                                                                   
  for( route::iterator i=last+1; i!=R->end(); i++) {                                                                                                            
   for( route::iterator j=R->begin(); j!=last; j++) {                                                                                                           
                                                                                                                                                                
    d[0]=std::abs(*j-*i)+std::abs(*i-*(j+1))-std::abs(*j-*(j+1));                                                                                               
    if(d[0]<d[1]) {                                                                                                                                             
     d[1]=d[0];                                                                                                                                                 
     closer[0]=i;                                                                                                                                               
     closer[1]=j;                                                                                                                                               
    }                                                                                                                                                           
   }                                                                                                                                                            
                                                                                                                                                                
   d[0]=std::abs(*last-*i)+std::abs(*i-*R->begin())-std::abs(*last-*R->begin());                                                                                
   if(d[0]<d[1]) {                                                                                                                                              
    d[1]=d[0];                                                                                                                                                  
    closer[0]=i;                                                                                                                                                
    closer[1]=last;                                                                                                                                             
   }                                                                                                                                                            
  }                                                                                                                                                             
                                                                                                                                                                
  d[2]+=d[1];                                                                                                                                                   
                                                                                                                                                                
  point aux=*closer[0];                                                                                                                                         
  R->erase(closer[0]);                                                                                                                                          
  R->insert(closer[1]+1, aux);                                                                                                                                  
 }                                                                                                                                                              
 return d[2];                                                                                                                                                   
}  
\end{verbatim}
}

\newpage

\subsection{Dois ótimo: 2-opt}

	Outra estratégia de aproximação é o \emph{Dois ótimo} ( \emph{two optimal}),
no qual percorre-se o conjunto procurando-se possiveis cruzamentos de rotas e
elimina-se estas.

\foreach \x in {1,2,3}{ 
\begin{figure}[!ht]
\centering
\begin{tikzpicture}
\begin{axis}[ width=8cm]
\addplot+[only marks, mark size=0.5pt,red] table {arg/k-opt\x.dat};
\addplot[mark size=0.3pt,blue] table {arg/k-opt\x.dat};
\addplot+[only marks, mark size=1pt, red] coordinates {
	(5.06721,3.94673) [2]
};
\end{axis}
\end{tikzpicture}
\caption{Pontos roteados pelo 2-opt (\x)}
\label{k-opt\x}
\end{figure}
}

\begin{description}
 \item[Vantagens] Possui um custo computacional inferior ao do \emph{vizinho
mais próximo}
 \item[Desvantages] Para alguns casos são necessárias algumas iterações, o que
no pior caso faz seu custo se equiparar ao do \emph{vizinho mais próximo}
\end{description}


{\small 
\begin{verbatim}
double opt2( route *R) {                                                                                                                                        
                                                                                                                                                                
 //! Tomar dois pares de pontos dijuntos                                                                                                                        
 route::iterator aux[2];                                                                                                                                        
 for( route::iterator i=R->begin(); next(i,2)!=R->end(); i++) {                                                                                                 
  aux[0]=next(i);                                                                                                                                               
  for( route::iterator j=next(aux[0]); next(j)!=R->end(); j++) {                                                                                                
   aux[1]=next(j);                                                                                                                                              
  //! verificar se a cruzamento de rotas                                                                                                                        
  if( cross(*i, *aux[0], *j, *aux[1]))                                                                                                                          
   rotate( aux[0], j);                                                                                                                                          
  }                                                                                                                                                             
  if(i==R->begin()) continue;                                                                                                                                   
                                                                                                                                                                
  //! executar para o último elemento                                                                                                                           
  route::iterator j=prev(R->end());                                                                                                                             
  aux[1]=R->begin();                                                                                                                                            
                                                                                                                                                                
  //! verificar se a cruzamento de rotas                                                                                                                        
  if( cross(*i,*aux[0],*j,*aux[1]))                                                                                                                             
   rotate( aux[0], j);                                                                                                                                          
 }                                                                                                                                                              
 return cust_route(*R);                                                                                                                                         
                                                                                                                                                                
} 
\end{verbatim}
}

\subsection{Clarke e Wright}

	Um dos algoritmos mais simples para obtenção de uma aproximação do
\emph{VRP} é o algoritmo de Clarke e Wright, que consiste em adicionar rotas de
veículos adjacêntes caso isto diminua seu custo.


	Iniciando-se com um conjunto $A$ de pontos. Tomando-se o deposito como
origem do sistema, ordena-se $A$ radialmente obtendo-se $A'$ , ou seja, pontos
em uma mesma reta que estão mais próximo da origim do sistema são considerados
menores e pontos como mesma distância da origem com ângulo maior são
considerados maiores.  Tomando-se cada ponto em $A'$ como uma rota unitária,
aplica-se para cada par de rotas adjacêntes uma união simulada, caso a nova rota
geral tenha custo melhor que a rota atual. Este processo contínua até que não
haja melhoria na rota atual.

	O calculo do custo do algoritmo pode ser realizado imaginado-se os pontos
como faixas em um circulo. Cada barra será o começo de uma rota e o inicío da
seguinte, portanto dados $n$ pontos e $k$ veículos devemos ter
	\[C_k^n = \frac{n\times\ldots\times(n-k+1)}{k!}\] 
	rotas possíveis. Devemos ainda adicionar ao custo computacional acima o
calculo da rota ótima para cada subrota, o que é indicado abaixo pelo algoritmo
do \emph{TSP}.

{\newcommand{\prox}[1]{\textrm{PROX}(#1)}
\newcommand{\tsp}[1]{\textrm{TSP}(#1)}
\newcommand{\cust}[1]{\textrm{CUSTO}(#1)}
\begin{algorithm}[H]
\caption{Clarke \& Wright}
\begin{algorithmic}
	\State $A \leftarrow $ ler de um arquivo
	\State $A' \leftarrow$ ordenar $A$
	\State $B \leftarrow$ alocar cada elemento de A em um veículo
	\State $d_1,d_0 \leftarrow $ custo da frota $B$
	\State $C \leftarrow B'$
	\While{críterio princial}
		\For{$i \in [0,\#B]\cap \mathbb{Z}$}
			\State $B' \leftarrow B-\{B(i)\cup\prox{B(i)}\}\cup\tsp{\{B(i)\}\cup\{\prox{B(i)}\}}$ 
			\State $d_2 \leftarrow \cust{B'}$
			\If{$d_2 < d_1$}
				\State $d_1 \leftarrow d_2$
				\State $C \leftarrow B'$
			\EndIf
		\EndFor
		\If{$d_1 < d_0$}
			\State $d_0 \leftarrow d_1$
			\State $B \leftarrow C$
		\EndIf
	\EndWhile
\end{algorithmic}
\end{algorithm}
,
	onde $\prox$ é o elemento seguinte a $B(i)$ na lista circular, $\tsp$ é o
algorithmo utilizado para aproximar-se o TSP e $\cust$ é o custo utilizado na
heurística vigente.
}

	No pior caso teremos o conjunto será formado apenas por um veículo, logo o
custo computacional de obtenção desta resposta é $O(n^4\times N)$, onde $N$ dependendo do
crítério utilizado para alocação de veículos nas rotas. Obviamente isto será
determinado em função do percurso desta rota. O resultado do algoritmo acima é
mostrado nas figuras \ref{clarke_wright1}, \ref{clarke_wright2}  e
\ref{clarke_wright3} .

\begin{figure}[!ht]
\centering
\foreach \x in {1,2,3}{ 
\begin{subfigure}{\textwidth}
\centering
\begin{tikzpicture}
\begin{axis}[ width=8cm ]
\addplot+[only marks, mark size=0.5pt, red] table {arg/mymethod\x.dat};
\addplot[mark size=0.3pt, blue] table {arg/mymethod\x.dat};
\addplot+[only marks, red] coordinates {
	(5.06721,3.94673) [2]
};
\end{axis}
\end{tikzpicture}
\caption{Pontos roteados pelo Clarke \& Wright (\x)}
\label{clarke_wright\x}
\end{subfigure}
}
\end{figure}

\newpage

	O critério empregado para a distribuição dos veículos foi a minimização do
custo quadrático, ou seja, que a soma dos quadrados dos custos de cada rota
fosse mínimo. 

Obs: o algoritmo em sí pode ser melhorado aplicando um realocamento de pontos já
pertencentes a uma rota adjacêntes.


{\small
\begin{verbatim}
double clarke_wright_tsp ( route* first,
	 train* second, double (*TSP)(route*))
{

 int N=first->size();

 point aux=(*first)[0];

 // Ordenar
 for( route::iterator i=first->begin(); i!=first->end(); i++) {
  *i-=aux;
 }

 std::sort(first->begin()+1, first->end(), operator<);

 for( route::iterator i=first->begin(); i!=first->end(); i++) {
  *i+=aux;
 }

// Alocar
 for( int i=1; i<N; i++) {
  route aux;
  aux.push_back((*first)[0]);
  aux.push_back((*first)[i]);
  second->push_back(aux);
  aux.clear();
 }

 double d[3]={0,0,cust_heuristic(*second)};

 train Aux, Aux1(*second);

 for ( int k=0; k<N-1; k++)
 {
  d[2]=d[1];
  Aux=train(*second);
  *second=train(Aux1);
  d[1]=cust_heuristic(Aux);
  for( train::iterator i=Aux.begin(); i+1!=Aux.end(); i++) {
   // Mesclar roca com a seguinte numa variável auxiliar
   i->insert(i->end(), (i+1)->begin()+1, (i+1)->end());
   // Aplicar o tsp
   (*TSP)(&*i);
   // Inserir rotas mescladas no train Aux
   Aux.erase(i+1);
   // Avaliar roteamento
   d[0]=cust_heuristic(Aux);
   // Ordenar custo
   if(d[0]<d[1]) {
    d[1]=d[0];
    Aux1=train(Aux);
   }
   Aux=train(*second);
  }

  Aux=train(*second);
  
  (Aux.end()-1)->insert((Aux.end()-1)->end(), 
			(Aux.begin())->begin(), (Aux.begin())->end());

  (*TSP)(&*(Aux.end()-1));
  // Inserir rotas mescladas no train Aux
  Aux.erase(Aux.begin());
  // Avaliar roteamento
  d[0]=cust_heuristic(Aux);
 // Ordenar custo
  if(d[0]<d[1]) {
   d[1]=d[0];
   Aux1=train(Aux);
  }
 }

 return cust_train(*second);
}
\end{verbatim}
}


\newpage


\subsection{Convexo}

	Para tentar diminuir o custo de obtenção de uma aproximação para o
\emph{VRP} utilizou-se como conjunto original o formado pelo feixo convexo dos
pontos adquiridos ( vide Figura \ref{conv1}).

	A seguir é mostrado o código principal do algoritmo do fecho convexo.

{\small
\begin{verbatim}
route bordo_convexo( route X ) {                                                                                                                                
                                                                                                                                                                
 route fecho;                                                                                                                                                   
 route AUX(X);                                                                                                                                                  
                                                                                                                                                                
 fecho.push_back(AUX[0]);                                                                                                                                       
 fecho.push_back(AUX[1]);                                                                                                                                       
                                                                                                                                                                
 AUX.erase(AUX.begin(), AUX.begin()+2);                                                                                                                         
                                                                                                                                                                
 for( uint k=2; k<AUX.size(); k++)                                                                                                                              
  if( sign(AUX[k]-fecho[0],fecho[1]-fecho[0]) >0) {                                                                                                             
   fecho.push_back(AUX[k]);                                                                                                                                     
   AUX.erase(AUX.begin()+k);                                                                                                                                    
   break;                                                                                                                                                       
  }                                                                                                                                                             
                                                                                                                                                                
 for( uint k=2; k<AUX.size(); k++) {                                                                                                                            
   if( In(fecho, AUX[k])) continue;                                                                                                                             
   uint v=positive( fecho, AUX[k]), w=negative( fecho, AUX[k]);                                                                                                 
   apagar(fecho, v, w);                                                                                                                                         
   if( w>v)                                                                                                                                                     
    fecho.insert(fecho.begin()+v+1, AUX[k]);                                                                                                                    
   else                                                                                                                                                         
    fecho.insert(fecho.begin(), AUX[k]);                                                                                                                        
  }                                                                                                                                                             
                                                                                                                                                                
 return fecho;                                                                                                                                                  
}
\end{verbatim}
}

\newpage

\begin{figure}[!ht]
\centering
\foreach \x in {1,2,3}{
\begin{subfigure}{\textwidth}
\centering
\begin{tikzpicture}
\begin{axis}[ width=8cm ]
\addplot+[only marks, mark size=0.3pt, red] table {arg/points\x.dat};
\addplot+[only marks, red] coordinates {
	(5.06721,3.94673) [2]
};
\addplot[mark size=0.1pt, blue] table {arg/convexo\x.dat};
\end{axis}
\end{tikzpicture}
\caption{Pontos roteados pelo fecho convexo e alocação radial (\x)}
\label{conv\x}
\end{subfigure}
}
\end{figure}

\newpage

\section{Conclusão}

	O estado posterior de implementação será em GPUs, pois estas tem uma
capacidade de processamento superior para certos algoritmos. Já foram
implementados alguns resultados relevântes nesta área, porem não serão
apresentados neste relatório devido a inexperiência com esta ferramenta.

\end{document}
