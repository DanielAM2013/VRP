\documentclass[a4paper, 12pt]{article}

\usepackage{preamble}
\usepackage{pgfplots}
\begin{document}

\thispagestyle{empty}

{\centering
{\sc 
\textbf{
Universidade Federal do Amazonas \\
Instituto de Computação\\
Projeto Promobile\\[4cm]
Daniel Antela Torquato\\[4cm]
Gerenciamento de rotas para multiplos veículos}}
\vfill
{\bf Manaus-AM\\2015\\}
}

\newpage


\section{Objetivos}

	Este projeto visa a criação e emprego de algoritmos capazes de coordenas de
forma eficiênte um conjunto de VANT's.

	Os principais problemas existentes são
\begin{description}
	\item[TSP] Problema do caixeiro viajante:

	Dado um conjunto de pontos pre-estabelecidos determinar a melhor rota que,
partindo de um destes, percorra todos os pontos apenas uma vez e volte a origem.

	\item[VRP] Problema de roteamento de veículos:

	Dado um conjunto de pontos pre-estabelecidos determinar o melhor conjunto de
rotas que, partindo de um mesmo ponto, percorram todos os pontos apenas uma vez
por qualquer veículo e volte a origem.
\end{description}

	A dificuldade para a resolução dos problemas acima não esta em descobrir um
método de resolução para uma determinada instância ( conjunto de pontos ), mas
em descobrir o procedimento que, em tempo hábil, determine uma solução
satisfatória.

\underline{Exemplo:}
	Considere um conjunto de pontos mapeados na Figura \ref{test}, a Figura
\ref{roteado} apresenta uma possivel solução para o TSP desta instância.


\begin{figure}[!ht]
\centering
\begin{tikzpicture}
\begin{axis}[
	xlabel={$x$},
	ylabel={$y$}
]
\addplot+[only marks] table {arg/points.dat};
\addplot+[only marks] coordinates {
	(5.06721,3.94673) [2]
};
\end{axis}
\end{tikzpicture}
\caption{Pontos de verificação}
\label{test}
\end{figure}
\begin{figure}[!ht]
\centering
\begin{tikzpicture}
\begin{axis}[
	xlabel={$x$},
	ylabel={$y$}
]
\addplot table {arg/output.dat};
\addplot+[only marks] coordinates {
	(5.06721,3.94673) [2]
};
\end{axis}
\end{tikzpicture}
\caption{Pontos roteados pelo vizinho mais próximo}
\label{roteado}
\end{figure}








\section{Estado atual}

	No atual estado de desenvolvimento já foram implementadas soluções simples
para execução em CPUs.

\section{Comentários}

	O estado posterior de implementação será em GPUs, pois estas tem uma
capacidade de processamento superior para certos algoritmos.


\section{Evidências}

%\input{lib/master.tex}
%\input{lib/2phase.tex}
%\input{lib/aprox_tsp.tex}

\end{document}
