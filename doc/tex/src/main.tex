\documentclass[a4paper, 12pt]{article}

\usepackage{preamble}
\usepackage{pgfplots}
\usepackage{verbatim}
\usepackage{algorithm}
\usepackage[noend]{algpseudocode}

\begin{document}

\thispagestyle{empty}

{\centering
{\sc 
\textbf{
Universidade Federal do Amazonas \\
Faculdade de Tecnologia\\
Projeto Promobile\\[4cm]
Daniel Antela Torquato\\[4cm]
Gerenciamento de rotas para multiplos veículos}}
\vfill
{\bf Manaus-AM\\2015\\}
}

\newpage


\section{Objetivos}

	Este projeto visa a criação e emprego de algoritmos capazes de coordenas de
forma eficiênte um conjunto de VANT's.

	Os principais problemas existentes são
\begin{description}
	\item[TSP] Problema do caixeiro viajante:

	Dado um conjunto de pontos pre-estabelecidos determinar a melhor rota que,
partindo de um destes, percorra todos os pontos apenas uma vez e volte a origem.

	\item[VRP] Problema de roteamento de veículos:

	Dado um conjunto de pontos pre-estabelecidos determinar o melhor conjunto de
rotas que, partindo de um mesmo ponto, percorram todos os pontos apenas uma vez
por qualquer veículo e volte a origem.
\end{description}

	A dificuldade para a resolução dos problemas acima não esta em descobrir um
método de resolução para uma determinada instância ( conjunto de pontos ), mas
em descobrir o procedimento que, em tempo hábil, determine uma solução
satisfatória.

\underline{Exemplo:}
	Considere um conjunto de pontos mapeados na Figura \ref{test}, a Figura
\ref{roteado} apresenta uma possivel solução para o TSP desta instância.


\begin{figure}[!ht]
\centering
\begin{tikzpicture}
\begin{axis}[
	xlabel={$x$},
	ylabel={$y$}
]
\addplot+[only marks] table {arg/points.dat};
\addplot+[only marks] coordinates {
	(5.06721,3.94673) [2]
};
\end{axis}
\end{tikzpicture}
\caption{Pontos de verificação}
\label{test}
\end{figure}
\begin{figure}[!ht]
\centering
\begin{tikzpicture}
\begin{axis}[
	xlabel={$x$},
	ylabel={$y$}
]
\addplot table {arg/output.dat};
\addplot+[only marks] coordinates {
	(5.06721,3.94673) [2]
};
\end{axis}
\end{tikzpicture}
\caption{Pontos roteados pelo vizinho mais próximo}
\label{roteado}
\end{figure}

	A solução apresentada não é a melhor existente porem foi obtida em uma
fração do tempo necessário para a obtenção da \emph{solução ótima}.


\section{Estado atual}

	No atual estado de desenvolvimento já foram implementadas soluções simples
para execução em CPUs.


\subsection{Vizinho mais próximo}

	Uma das heurísticas mais clássicas para obter-se uma aproximação para o TSP
consiste no algoritmo do vizinho mais próximo, o qual e descrito abaixo.

% R <- A_0
% while not_empty(A)
% [p,q]=point_closer(R,A)
% add p to B in q
% remove(p,B);
% end while



\begin{algorithmic}
 \State $S \leftarrow 0$
 \While{ Se A é não vazio }
  \State $ [p,q] \leftarrow$ pontos mais próximos entre $R$ e $A$,
respectivamente.
  \State adicionar $p$ a $B$ em $q$
 \EndWhile
\end{algorithmic}

	Para descobrir os pontos mais próximos entre $R$ e $A$ deve-se utiliza o
seguinte algoritmo

\begin{algorithmic}
	\For{$i$ em $A$}
		\For{$j$ em $B$}
			\State $d_2 \leftarrow $ distância entre $A_i$ e $B_j$
			\If{$d_2 < d_1$}
				\State $d_1 \leftarrow d_2$
				\State $\textrm{indice}_1 \leftarrow i$
				\State $\textrm{indice}_2 \leftarrow j$
			\EndIf
		\EndFor
	\EndFor
\end{algorithmic}

	O algoritmo acima tem um custo computacional de $n\times m$, onde $n$ é o
número de termos em $A$ e $m$ é o número de termos em $B$. No algoritmo inicial
teremos que utilizar o algoritmo o algoritmo de \emph{pontos mais próximos} até
que o conjunto $A$ seja vazio. A cada passo pelo menos um ponto será retirado de
$A$ e adicionado a $B$ o que significa que termos $n$ passos. Além disso no
passo inicial $B$ tem apenas um ponto enquanto $A$ tem $n-1$. No $k$-esimo passo
$B$ terá $k$ pontos enquanto $A$ terá $n-k$ pontos, logo o algoritmo de
\emph{pontos mais próximos} terá um custo de $k\times(n-k)$ operações, logo o
algoritmo completo executará $\sum_{k=1}^{n} k\times (n-k)$ operações que é
igual à $\frac{n\times(n+1)\times(2n+1)}{3}$ dizemos neste caso que o algoritmo
é de ordem $O(n^3)$ pois seu fator dominante para grandes números será $n^3$


\subsection{Clarke e Wright}

	Um dos algoritmos mais simples para obtenção de uma aproximação do
\emph{VRP} é o algoritmo de Clarke e Wright, que consiste em adicionar rotas de
veículos adjacêntes caso isto diminua seu custo.


	Iniciando-se com um conjunto $A$ de pontos. Tomando-se o deposito como
origem do sistema, ordena-se $A$ radialmente, ou seja, pontos em uma mesma reta
que estão mais próximo da origim do sistema são considerados menores e pontos
como mesma distância da origem com ângulo maior são considerados maiores.
Tomando-se cada ponto em $A'$ como uma rota unitária, aplica-se para cada par de
rotas adjacêntes uma união simulada, caso a nova rota geral tenha custo melhor
que a rota atual. Este processo contínua até que não haja melhoria na rota
atual.

	O calculo do custo do algoritmo pode ser realizado imaginado-se os pontos
como faixas em um circulo. Cada barra será o começo de uma rota e o inicío da
seguinte, portanto dados $n$ pontos e $k$ veículos devemos ter
	\[C_k^n = \frac{n\times\ldots\times(n-k+1)}{k!}\] 
	rotas possíveis. Devemos ainda adicionar ao custo computacional acima o
calculo da rota ótima para cada subrota.

{\newcommand{\prox}[1]{\textrm{proximo}(#1)}
\newcommand{\tsp}[1]{\textrm{tsp}(#1)}
\newcommand{\cust}[1]{\textrm{custo}(#1)}
\begin{algorithm}[H]
\caption{Clarke \& Wright}
\begin{algorithmic}
	\State $A \leftarrow $ ler de um arquivo
	\State $A' \leftarrow$ ordenar $A$
	\State $B \leftarrow$ alocar cada elemento de A em um veículo
	\State $d_1,d_0 \leftarrow $ custo da frota $B$
	\State $C \leftarrow B'$
	\While{críterio princial}
		\For{$i \in [0,\#B]\cap \mathbb{Z}$}
			\State $B' \leftarrow B-\{B(i)\}-\{\prox{B(i)}\}\cup\tsp{\{B(i)\}\cup\{\prox{B(i)}\}}$ 
			\State $d_2 \leftarrow \cust{B'}$
			\If{$d_2 < d_1$}
				\State $d_1 \leftarrow d_2$
				\State $C \leftarrow B'$
			\EndIf
		\EndFor
		\If{$d_1 < d_0$}
			\State $d_0 \leftarrow d_1$
			\State $B \leftarrow C$
		\EndIf
	\EndWhile
\end{algorithmic}
\end{algorithm}
}

	No pior caso teremos o conjunto será formado apenas por um veículo, logo o
custo computacional de obtenção desta resposta é $O(n^4\times N)$, onde $N$ dependendo do
crítério utilizado para alocação de veículos nas rotas. Obviamente isto será
determinado em função do percurso desta rota. O resultado do algoritmo acima é
mostrado na figura abaixo.

\begin{figure}[!ht]
\centering
\begin{tikzpicture}
\begin{axis}[
	xlabel={$x$},
	ylabel={$y$}
]
\addplot table {arg/output2.dat};
\addplot+[only marks] coordinates {
	(5.06721,3.94673) [2]
};
\end{axis}
\end{tikzpicture}
\caption{Pontos roteados pelo Clarke \& Wright}
\label{roteado}
\end{figure}


	O critério empregado para a distribuição dos veículos foi a minimização do
custo quadrático, ou seja, que a soma dos quadrados dos custos de cada rota
fosse mínimo. 

Obs: o algoritmo em sí pode ser melhorado aplicando um realocamento de pontos já
pertencentes a uma rota adjacêntes.


\subsection{Convexo}

	Para tentar diminuir o custo de obtenção de uma aproximação para o
\emph{TSP}

\begin{figure}[!ht]
\centering
\begin{tikzpicture}
\begin{axis}[
	xlabel={$x$},
	ylabel={$y$}
]
\addplot+[only marks] table {arg/points.dat};
\addplot+[only marks] coordinates {
	(5.06721,3.94673) [2]
};
\addplot table {arg/output3.dat};
\end{axis}
\end{tikzpicture}
\caption{Pontos roteados pelo Clarke \& Wright}
\label{roteado}
\end{figure}

\section{Comentários}

	O estado posterior de implementação será em GPUs, pois estas tem uma
capacidade de processamento superior para certos algoritmos. Já foram
implementados alguns resultados relevântes nesta área, porem não serão
apresentados neste relatório.


\section{Evidências}

	A seguir é apresentado o código de construção do algoritmo para o feixo
convexo de um conjunto de pontos 


{\small
\verbatiminput{../../src/Convexo/src/convexo.cpp}
}



\end{document}
