\documentclass[a4paer, 12pt]{article}
%	Codigo de Caracteres
\usepackage[utf8]{inputenc}

%	Correção de palavras
\usepackage[portuguese]{babel}
\usepackage[T1]{fontenc}

%	Matematico
\usepackage{amsmath}
\usepackage{amsfonts}
\usepackage{amssymb}
\usepackage{makeidx}

%	Grafico
\usepackage{graphicx,color}


\usepackage{wrapfig}
\usepackage{setspace}

%	Formatação
\usepackage[left=3cm,right=3cm,top=3cm,bottom=2cm]{geometry}
\usepackage{hyperref}
\usepackage{graphpap}



%	Desenho
\usepackage{pst-plot}
\usepackage{pstricks}
\usepackage{xcolor-patch,pict2e,curve2e}
\usepackage{epic}
%\usepackage{pstbpdf}

%	Sumario

\usepackage{tocloft}

%	Custom

\newtheorem{definicao}{Definição}
\newtheorem{conceito}{Conceito}

\newtheorem{exemplo}{Exemplo}

\newtheorem{proposicao}{Proposição}
\newtheorem{teorema}{Teorema}
\newtheorem{lema}{Lema}
\newtheorem{prova}{Prova}

\newenvironment{dem}{\flushleft \textbf{ Demonstração:}}{ \hfill $\square$}

\newenvironment{ex}{\flushleft \textbf{Exemplo}.}{ \hfill $\lozenge$}

\newenvironment{res}{\flushleft \underline{ \textbf{Resolução:}}}{ \hfill $\triangleleft$}

\newenvironment{midpage}{\vspace*{\fill}}{\vspace*{\fill}}


\begin{document}

 Umas das aproximações exatas mais utilizadas para CVRP é o método da $k$-árvore de {\color{red}
referência} que conseguiu resolver o problema para 71 clientes. Contudo, há muitas pequenas senteças
que não tem já sido resolvidas exatamente. Para tratar de grandes instâncias, ou para calcular
soluções rapidamente, métodos heuristicos devem ser usados. Usa um novo ramo e procedimento
limitado no qual o problema é particionado são dualizados para obter uma relaxamento Lagrangeano que
é um grau de contração mínimo para o problema de $k$-árvore.

 A aproximação de $k$-árvore pode ser extendida para acomodar variações realisticas, tais como custo
assimétrico, janela de tempo, e vôos não uniformes.

 Um algoritmo de ramo e limite usa uma estratégia de divisória e conquistar para particionar o
espaço de solução $S$. No processo ou limitando fase, inicialmente examinamos o espaço de solução
$S$ inteiramente. Na fase limitação ou processamento relaxamos o problema. Então fazemos, admitimos
qua a solução não estão no conjunto possível $S$. Resolvemos o problema relaxado produzindo um
limite inferior nos valores da solução ótima. Se a solução para esta relaxação é um membro de $S$ ou
tem um custo igual algum $\tilde{s}$


\end{document}
