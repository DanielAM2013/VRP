\documentclass[a4paper,12pt]{article}
%	Codigo de Caracteres
\usepackage[utf8]{inputenc}

%	Correção de palavras
\usepackage[portuguese]{babel}
\usepackage[T1]{fontenc}

%	Matematico
\usepackage{amsmath}
\usepackage{amsfonts}
\usepackage{amssymb}
\usepackage{makeidx}

%	Grafico
\usepackage{graphicx,color}


\usepackage{wrapfig}
\usepackage{setspace}

%	Formatação
\usepackage[left=3cm,right=3cm,top=3cm,bottom=2cm]{geometry}
\usepackage{hyperref}
\usepackage{graphpap}



%	Desenho
\usepackage{pst-plot}
\usepackage{pstricks}
\usepackage{xcolor-patch,pict2e,curve2e}
\usepackage{epic}
%\usepackage{pstbpdf}

%	Sumario

\usepackage{tocloft}

%	Custom

\newtheorem{definicao}{Definição}
\newtheorem{conceito}{Conceito}

\newtheorem{exemplo}{Exemplo}

\newtheorem{proposicao}{Proposição}
\newtheorem{teorema}{Teorema}
\newtheorem{lema}{Lema}
\newtheorem{prova}{Prova}

\newenvironment{dem}{\flushleft \textbf{ Demonstração:}}{ \hfill $\square$}

\newenvironment{ex}{\flushleft \textbf{Exemplo}.}{ \hfill $\lozenge$}

\newenvironment{res}{\flushleft \underline{ \textbf{Resolução:}}}{ \hfill $\triangleleft$}

\newenvironment{midpage}{\vspace*{\fill}}{\vspace*{\fill}}

\usepackage[]{algorithm2e}

\begin{document}
\section{begin}

 Um algoritmo interessante para se iniciar a implementação de métodos de aproximação da solução do
VRP é o método de clusterização seguido de roteamente.

\section{Ponto inicial}

 Dado um conjunto de ponto tomamos como rotas inicias os pares adjacentes de retas que se cruzam no
deposito, de modo que o ultimo ponto de uma reta esteja ligado ao último ponto da outra formando uma
rota até o depósito. 

\section{Critérios de Mesclagem}

 Dada a imensidão de possibilidades de posicionamento de um ponto no conjunto que forma o grafo fica
difícil determinar um único método de mesclagem de rotas.

 Sobre certas condições pode ser mais proveitoso deslocar um ponto de uma rota para outra adjacente
, dividir os pontos de uma rota entre suas duas vizinhas ou simplemente parar.

 Para decidir o que deve ser feito para garantir um processor de melhorias incrementais precisamos
explorar 

 O processo de pesquisa 



\section{Minha versão do Algoritmo de Clack e Wright}

{\renewcommand{\labelenumi}{Step \arabic{enumi}}
\begin{enumerate}
\item Inicialização: tomar como rotas iniciar os pontos isolados
\item Varredura: varrear todas as rotas adjacentes calculando a economia proporcionada por sua
mescla e tomar a com maior economia.
Obs: o custo de uma mescla será a norma do $\mathbb{R}^n$( para n pontos), onde cada coordenada é
uma subrota.
\item Parar quando o custo não diminuir mais.
\end{enumerate}



<<<<<<< HEAD



<<<<<<< HEAD
=======
=======
>>>>>>> 94313e695f6f0af351c97b0573b10a4e55f18438
\section{TSP}

\subsection{Vizinho mais próximo}

\begin{algorithm}[!ht]
 \KwData{ P}
 \KwResult{O}
 initialization\:

 \While{ x $\neq$ 2}{
  read current\;
  \eIf{understand}{
   go to next section\;
   current section becomes this one\;
   }{
   go back to the beginning of current section\;
  }
 }
 \caption{How to write algorithms}
\end{algorithm}





<<<<<<< HEAD

























>>>>>>> 5181573238744e8e2018ff3e54ad6d6c4d4bd077





=======
>>>>>>> 94313e695f6f0af351c97b0573b10a4e55f18438
\end{document}
