\documentclass[a4paper, 12pt]{article}
%	Codigo de Caracteres
\usepackage[utf8]{inputenc}

%	Correção de palavras
\usepackage[portuguese]{babel}
\usepackage[T1]{fontenc}

%	Matematico
\usepackage{amsmath}
\usepackage{amsfonts}
\usepackage{amssymb}
\usepackage{makeidx}

%	Grafico
\usepackage{graphicx,color}


\usepackage{wrapfig}
\usepackage{setspace}

%	Formatação
\usepackage[left=3cm,right=3cm,top=3cm,bottom=2cm]{geometry}
\usepackage{hyperref}
\usepackage{graphpap}



%	Desenho
\usepackage{pst-plot}
\usepackage{pstricks}
\usepackage{xcolor-patch,pict2e,curve2e}
\usepackage{epic}
%\usepackage{pstbpdf}

%	Sumario

\usepackage{tocloft}

%	Custom

\newtheorem{definicao}{Definição}
\newtheorem{conceito}{Conceito}

\newtheorem{exemplo}{Exemplo}

\newtheorem{proposicao}{Proposição}
\newtheorem{teorema}{Teorema}
\newtheorem{lema}{Lema}
\newtheorem{prova}{Prova}

\newenvironment{dem}{\flushleft \textbf{ Demonstração:}}{ \hfill $\square$}

\newenvironment{ex}{\flushleft \textbf{Exemplo}.}{ \hfill $\lozenge$}

\newenvironment{res}{\flushleft \underline{ \textbf{Resolução:}}}{ \hfill $\triangleleft$}

\newenvironment{midpage}{\vspace*{\fill}}{\vspace*{\fill}}


\begin{document}

\section{Algoritmo Formiga}

 O primeiro sistema formiga para VRP foi desenhado muito recentemente por {\color{red} referência},
o qual considera a versão mais elementar do problema: CVRP.

 Para mais versões mais complexas de VRP, {\color{red} referência} tem desenvolvido um sistema de
multiplas colonias de formigas para VRPTW (MACS-VRPTW) o qual é organizado com uma hierarquia de
desenho de colonias artificais de formigas para sucessivamente otimizar uma função de multiplos
objetivos: a primeira colonia minimiza o numero de veículos enquanto a segunda colonia minimiza a
distância de viagem. Cooperação entre colonias é feita mudando informações através de atualização de
feromônio.

 Em {\color{red} referência} desenho, há duas fases básicas de sistemas de formigas: construção de
veículos e atualização de trilha, O Algoritmo AS é explicado aqui.

\subsection{ Algoritmo de Sistema de formigas}

 Após inicializar o AS, os dois passos básicos de construção de rotas de veículos e atualização de
trilha são repetidos para um número de iterações. Considerando a disposição inicial das formigas
artificiais foi encotrado que o número de formigas poderia ser igual em cada cliente no inicio da
iteração. O 2-opt-heuritica ( é explicada esplorando todas as permutações obtidas pela mudança de
duas cidades) é usando para minimizar a rota de veículos geredas pelas formigas artificiais,
consideravelmente melhorar a qualidade da solução. Além disso para esse avançar na pesquisa local
tambem introduzimos uma listade de cadidatos para seleção de clientes os quais são determinados na
fase de inicialização do algoritmo. Para cada localição $d_{ij}$ ordenamos $V-\{v_i\}$. de acordo
com a distância crescente $d_{ij}$ para obter a lista de candidatos. A proposição de AS para CVRP
pode ser descrita pelo seguinte algoritmo esquematico:

\begin{enumerate}
\item Inicialize
\item Para $I^\max$ iterações faça:

\begin{enumerate}
\item Para todas as formigas gerar um nova solução usando a Fórmula 1 e a lista de candidatos
\item Melhorar todas as rotas de veículos usando o 2-opt-heuristica
\item Atualizar as trilhas de feromônio usando Fórmula 2.
\end{enumerate}
\end{enumerate}

\subsection{





\end{document}
