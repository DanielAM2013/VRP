\documentclass[a4paper, 12pt]{article}
%	Codigo de Caracteres
\usepackage[utf8]{inputenc}

%	Correção de palavras
\usepackage[portuguese]{babel}
\usepackage[T1]{fontenc}

%	Matematico
\usepackage{amsmath}
\usepackage{amsfonts}
\usepackage{amssymb}
\usepackage{makeidx}

%	Grafico
\usepackage{graphicx,color}


\usepackage{wrapfig}
\usepackage{setspace}

%	Formatação
\usepackage[left=3cm,right=3cm,top=3cm,bottom=2cm]{geometry}
\usepackage{hyperref}
\usepackage{graphpap}



%	Desenho
\usepackage{pst-plot}
\usepackage{pstricks}
\usepackage{xcolor-patch,pict2e,curve2e}
\usepackage{epic}
%\usepackage{pstbpdf}

%	Sumario

\usepackage{tocloft}

%	Custom

\newtheorem{definicao}{Definição}
\newtheorem{conceito}{Conceito}

\newtheorem{exemplo}{Exemplo}

\newtheorem{proposicao}{Proposição}
\newtheorem{teorema}{Teorema}
\newtheorem{lema}{Lema}
\newtheorem{prova}{Prova}

\newenvironment{dem}{\flushleft \textbf{ Demonstração:}}{ \hfill $\square$}

\newenvironment{ex}{\flushleft \textbf{Exemplo}.}{ \hfill $\lozenge$}

\newenvironment{res}{\flushleft \underline{ \textbf{Resolução:}}}{ \hfill $\triangleleft$}

\newenvironment{midpage}{\vspace*{\fill}}{\vspace*{\fill}}


\begin{document}

\section{Algoritmo Formiga}

 O primeiro sistema formiga para VRP foi desenhado muito recentemente por {\color{red} referência},
o qual considera a versão mais elementar do problema: CVRP.

 Para mais versões mais complexas de VRP, {\color{red} referência} tem desenvolvido um sistema de
multiplas colonias de formigas para VRPTW (MACS-VRPTW) o qual é organizado com uma hierarquia de
desenho de colonias artificais de formigas para sucessivamente otimizar uma função de multiplos
objetivos: a primeira colonia minimiza o numero de veículos enquanto a segunda colonia minimiza a
distância de viagem. Cooperação entre colonias é feita mudando informações através de atualização de
feromônio.

 Em {\color{red} referência} desenho, há duas fases básicas de sistemas de formigas: construção de
veículos e atualização de trilha, O Algoritmo AS é explicado aqui.

\subsection{ Algoritmo de Sistema de formigas}

 Após inicializar o AS, os dois passos básicos de construção de rotas de veículos e atualização de
trilha são repetidos para um número de iterações. Considerando a disposição inicial das formigas
artificiais foi encotrado que o número de formigas poderia ser igual em cada cliente no inicio da
iteração. O 2-opt-heuritica ( é explicada esplorando todas as permutações obtidas pela mudança de
duas cidades) é usando para minimizar a rota de veículos geredas pelas formigas artificiais,
consideravelmente melhorar a qualidade da solução. Além disso para esse avançar na pesquisa local
tambem introduzimos uma listade de cadidatos para seleção de clientes os quais são determinados na
fase de inicialização do algoritmo. Para cada localição $d_{ij}$ ordenamos $V-\{v_i\}$. de acordo
com a distância crescente $d_{ij}$ para obter a lista de candidatos. A proposição de AS para CVRP
pode ser descrita pelo seguinte algoritmo esquematico:

\begin{enumerate}
\item Inicialize
\item Para $I^{\max}$ iterações faça:

\begin{enumerate}
\item Para todas as formigas gerar um nova solução usando a Fórmula 1 e a lista de candidatos
\item Melhorar todas as rotas de veículos usando o 2-opt-heuristica
\item Atualizar as trilhas de feromônio usando Fórmula 2.
\end{enumerate}
\end{enumerate}

\subsection{Construção das rotas de veículos}

 Para resolver o VRP, as formigas artificiais constrões soluções por escolhas sucessivas de cidades
para visitar, até que cada cidade tenha sido visitada. Sempre que a escolha de outra cidade
liderarsse um solução impossivel por razões de capacidade de veículos ou comprimento total de rota,
o deposito é escolhido e uma nova rota é iniciada. Para seleção de uma cidade, dois aspectos são
levados em conta: qual bom foram as escolhas desta cidade, um informação que é quardada nas trilhas
de feromônio $\tau_{ij}$ é associada com cada arco $(v_i, v_j)$, e qual promisora é a escolha desta
cidade. Essa última medida de desejo, chamada visibilidade e denotada po $\eta_{ij}$, é a função de
heuristica local mensionada acima.

 Com $\Omega = \{ v_i \in V: v_j\textrm{ é possivel ser visitado} \} \cup \{v_0\}$, a cidade $v_j$ é
selecionada para ser visitada com segue:

\[p_{ij} = \left\{
\begin{array}{ll}
\frac{[\tau_{ij}]^\alpha[\eta_{ij}]^\beta}{\sum_{k\in\Omega[\tau_{ij}]^\alpha[\eta_{ij}]^\beta}}
& \textrm{if } v_j \in \Omega\\
0 & \textrm{caso contrário}
\end{array}\right.\]

 Essa possibilidade de distribuição é inclinada pelos parâmetros $\alpha$ e $\beta$ que determinan a
influência relativa das trilhas e a vizibilidade, respectivamente. A vizibilidade é determinada como
a recíproca da distância, e a probabilidade de seleção é então extendida pela informação específica
do problema. Aqui, a inclusão de poupaças e capacidade utilizam a liderança para melhorar
resultados. De outra forma, o último é relativamente custoso em termos de tempo computacional e
assim não usaremos neste paper. Assim, introduziremos os parâmetros $f$ e $g$, e usaremos os
seguintes função paramétrica de economia para a visibilidade:

 \[\eta_{ij} = d_{i0}+d_{0j}-gd_{ij}+f|d_{i0}-d_{0j}|\]

\subsection{Atualização de trilha}

 Após uma formiga artificial ter construido uma solução possivel, as trilhas de feromônio são
sulcado(?) dependedo do valor objetico da solução. Essa regra de atualização é como segue:

\[\tau^{new}_{ij} = p \tau^{old}_{ij} + \sum_{\mu=1}^{\sigma-1} \Delta\tau^{\mu}_{ij}+\sigma
\Delta\tau^{*}_{ij}\]

 onde $p$ é a pesistência da trilha (com $0 \leq \rho \leq 1$, assim a evaporação da trilha é dada
por $(1-\rho)$. Únicamente se o arco $(v_i, v_j)$ foi usado pela $\mu$-esima melhor formiga, a
trilha de feromônuo é incrementada em $\Delta \tau_{ij}^\mu$ o qual é então igual a $(\sigma -
\mu)/L_\mu$, e $0$ caso contrário. Além disso para que, todos os arcos pertençam a melhor solução
é enfatizado com se $\sigma$ formiga elitista fossem usadas então. Assim, cada formiga elitista
incrementa a intesidade da trilha em $\Delta \tau_{ij}^*$ que é igual a $1/L^*$ se o arco $(v_i,
v_j)$ pertence a melhor solução, e $0$ caso contrário.


\section{Algoritmo de programação de restrições}

 A programação de restrições (CP) {\color{red} referência} é um paradigma para representar e
resolver uma larga variedade de problemas. Problemas são expressos em termos de variáveis, domínios
para estas variáveis e restrições entre as variáveis. Os problemas são então resolvidos usando
tecnicas de pesquisa completa rais como depth-first search e branch e bound. A riqueza de linguagem
usada para expressar problemas em CP faz este um candidato ideal para VRPs. Isso podessibilida a
utilização de expressões mais gerais. Além disso expressões envolvendo a aritimética usual e
operações lógicas, restrições simbólicas complexas podem ser usadas para descrever problemas. CP
melhora a pesquisa usando propagação de restrições. Se os limites ou restrições de uma variável
podem ser inferidos, ou são conjuntos testaveis, essas mudanças são "propagadas" através de todas as
restrições para reduzir o domínio de variáveis restritoras.

 Em cada nó na arvore de pesquisa, os mecanismos de propagação removem valores do domínio de
variáveis restritoras que são inconsistentes com outras variáveis. Se o mecanismo de propagação
remove todos os valores de uma variável, então não pode haver um solução nesta sub-árvore e a
pesquisa de rastros faz uma decisão diferente no ponto de rastro. Rastreio é cronológigo: decisões
podem somente ser desfeitas em ordem oposta a qual elas foram feitas. Contudo, os domínios as
variáveis de restrição podem somente ser reduzidos com uma pesquisa descendete na árvore. O domínio
de todas as variáveis de restriçã pode ser restaurado pelo rastro para um nó anterior, mas
geralmente o aumento de domínios não é suportado. Isso tem implicações importantes para os meiors
nos quais o VRP é resolvido.

 No caso particular de resolver VRP, uma variável de decisão $R_i$ é associada com cada cliente
vizitado $i$, representando a próxima vizita feita pelo mesmo veículo. Para cada veículo $k$, há
visitas adicionais $S_k$ fazendo o inicio a rota, e $E_k$ fazendo o fim da rota. Para ler o fim do
intinerário para um veículo $k$, iniciado em $S_K$ e seguindo o próximo ponteiro através de $E_k$.

 O conjunto de clientes visitados será referenciado com $N$, a visita inicial com $S$, e a visita
final com $E$. $A =_{def} N\cup S\cup E$ são todas as visitas. Observando $R$ como uma função, $R$
aplica $N\cup S$ sobre $N\cup E$.

 Modelar VRP é desejavel ter um tipo de restrição especial para distribuir restrições ao longo dos
caminhos, As restrições são da forma $R_i = j \Rightarrow Q_j \geq Q_i + q_{ij}.$  

 Assim, se o visitante $j$ segue imediatamente o visitante $i$, a quantidade $Q$ é acumulada. Por
exemplo, tomando-se $q_{ij}$ iqual a o tempo de viágem entre $i$ e $j$ isto seria a o tempo de
chegada em $j$. Tomando-se $q_{ij}$ como a demanda em $i$ que seria acumulada no veículo. Isso
significa permitir outra restrição do mundo real ser expressa sucintamente.

 Em cada caso um ponto fixo deve ser suplimido, por exemplo $Q_i = 0 \forall i\in S$ no caso de
restrições de carga. Uma simples restrição deste tipo o qual restringe a limite superior tambem deve
ter os efeitos de eliminação de subrotas.

\subsection{Restrições para VRP}

\subsection{Núcleo de restrições}

\begin{itemize}
\item Time: é restrito pelo dia de trabalho e pela tempo de entrega do cliente.
\item Capacity: pode ser restrito em termos de peso, volume, número de lugares de pallet, etc.
\end{itemize}






\end{document}
