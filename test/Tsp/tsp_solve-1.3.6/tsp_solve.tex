\input texinfo.tex
@c %**start of header
@setfilename tsp_solve.info
@settitle GNU tsp_solve
@setchapternewpage odd
@c %**end of header

@iftex
@finalout
@end iftex

@ifinfo
This file documents the GNU tsp_solve program.

Copyright (C) 1992 Free Software Foundation, Inc.
Program and Manual authored by Chad Hurwitz.

Permission is granted to make and distribute verbatim copies of this
manual provided the copyright notice and this permission notice are
preserved on all copies.

@ignore
Permission is granted to process this file through Tex and print the
results, provided the printed document carries copying permission notice
identical to this one except for the removal of this paragraph (this
paragraph not being relevant to the printed manual).
@end ignore

Permission is granted to copy and distribute modified versions of this
manual under the conditions for verbatim copying, provided also that the
GNU Copyright statement is available to the distributee, and provided
that the entire resulting derived work is distributed under the terms of
a permission notice identical to this one.

Permission is granted to copy and distribute translations of this manual
into another language, under the above conditions for modified versions.
@end ifinfo

@titlepage
@title GNU tsp_solve
@subtitle Edition 1.3, for GNU tsp_solve version 1.4
@subtitle September 1994
@author by Chad Hurwitz

@page
@vskip 0pt plus 1filll
Copyright @copyright{} 1990, 1991, 1992 Free Software Foundation, Inc.

Permission is granted to make and distribute verbatim copies of
this manual provided the copyright notice and this permission notice
are preserved on all copies.

Permission is granted to copy and distribute modified versions of this
manual under the conditions for verbatim copying, provided that the entire
resulting derived work is distributed under the terms of a permission
notice identical to this one.

Permission is granted to copy and distribute translations of this manual
into another language, under the above conditions for modified versions,
except that this permission notice may be stated in a translation approved
by the Foundation.
@end titlepage

@node Top
@top

@ifinfo
This document describes the GNU tsp_solve program, a utility for obtaining
solutions to Traveling Salesperson Problems.

@end ifinfo

@menu
* Introduction::       TSPs and tsp_solve
* Using tsp_solve::    Arguments and Options of tsp_solve
* Tour Finders::       Tour Finder algorithms
* Tour Improvements::  Tour Improvement algorithms
* Examples::           Examples of usage
* Messages::           Error messages tsp_solve reports
* Notes::              How to determine if tsp_solve is what you need
* Bugs::               Caveats, Beetles
* Contributors::       The hall of champions
@end menu

@node Introduction
@chapter Introduction
The Traveling Salesperson Problem (TSP) is one of the seminal NP complete
problems that select mathematical fields become consumed over when they are
possessed to solve.  A TSP decides how to travel N cities exactly once while
minimizing the total cost it takes to travel between each city and returning
to the starting city.  The problem may be specified by giving coordinates of
the N cities, where the cost to travel each city is designated by the
euclidean distance between each city's coordinate location.  The problem may
also be specified by a matrix where each city has a row and column of costs
from and to each city.  A solution of a TSP is the tour that the imaginary
salesperson takes, i.e. an ordered permutation of the cities in the TSP.

tsp_solve can take a TSP as input, find a solution, and output the solution.
The solution found can be an optimal TSP solution or it can be a heuristic
solution depending on what kind of tour (solution) finder is specified.

@node Using tsp_solve
@chapter Using tsp_solve

@menu
* Input Options::      Input options of tsp_solve
* Output Options::     Output options of tsp_solve
* Advanced Options::   Less intuitive options
* Signal Responses::   What happens when a tsp_solve process is signaled
@end menu

@node Input Options
@section Input options of tsp_solve

TSPs can be input in one of three ways:

@enumerate
@item
automatically randomly generated

@item
input by matrix

@item
input by 2-D coordinates
@end enumerate

In circumstances where the input file does not describe the size of the TSP,
the number of cities must be specified on the command line as a positive
integer N.  With no input options, tsp_solve will generate a set of 2-D cities
and the cost matrix calculated by the euclidean distance between the city
coordinates: defaulting to '-m1'.

All numbers/values expected from user input can be separated by any character
besides the '.', 'e', 'E', '+', '-', or digits 0 through 9.  Exponentials and
negative values may be used as input.  Any alpha strings separated by
white space (including characters '.', 'e', 'E', '+', '-') that do not contain
digits will be ignored as separator space for the next value.

@table @samp
@item -m1
Generate random 2-D coordinates.

@item -m2
Generate a random symmetric TSP.

@item -m3
Generate a random asymmetric TSP.

@item -m4
Generate a random asymmetric oneway TSP.  see `-m-3'.

@item =
@itemx -m-1
Expect a list of N coordinate pairs from standard input.

@item -
@itemx -m0
Expect a full matrix to be passed from standard input.

@item -=
@itemx -m-2
Like the '=' option except expect an ordered list of integer city ids after the
coordinate list to specify what order the matrix should be filled from the
coordinates.  This is useful when the "easytour" tour finder is specified.  If
you know the order of cities but you don't know the order of coordinates and/or
you wish to find out the cost of that tour, cat the coordinate file with the
ordered set of city ids appended into a pipe "tsp_solve -= easy" with whatever
printing options needed.

@item -m-3
Expect a special one-way TSP as a series of numbers:  the number of cities (N),
followed by N costs representing the starting city to all N cities, followed by
N costs representing each of the N cities to the ending city, followed by an
NxN matrix as expected for the '-' option.  Note: this creates a TSP of N+2
cities where the starting city (N+1) is always traveled after the ending city
(N+2).  It is assumed, by assigning very large costs to the edges, impossible to
get from any N cities to the starting city and it is impossible to get from the
ending city to any other city except the staring city.

@item --
@itemx -m-4
Expect a .tsp file in TSPLIB format.  Since the number of cities is specified
in the TSPLIB file the command line Ncities will be ignored.  Also, if a
TOUR_SECTION which follows the TSPLIB file, the TSP matrix will be ordered by
that matrix.  Appending the TOUR_SECTION to the TSPLIB file and using the
"easytour" tour finder is a way to compute the "cost" of a TSPLIB solution
and/or output the TSPLIB solution in gnuplot format.

@itemx -m-5
Expect a symmetric one-way TSP.  Like the -m-3 option but all the costs of the
N cities should form a symmetric matrix.  The starting and ending (N+1 and N+2
respectivley) cities must be traveled adjacently though not in any directed
order.

@itemx =.
@itemx -m-6
expect a set of coordinates as in `=', but followed by an extraneous coordinate
Ef two numbers.  This option is most usefull when trying to read the `-p1'
gnuplot output format, which needs the extra coordinate to complete the plot
of the tour.

@item -fX
Transfer all expectations of input from standard input to the file 'X'.
@end table

Definitions: A TSP input by 2-D coordinates is called a Euclidean TSP.  A TSP
input from a symmetric cost matrix is called a Symmetric TSP.  A TSP input
with an asymmetric cost matrix is called an Asymmetric TSP.

@node Output Options
@section Output Options

The first line of all output, except when -p1 is used, will output a replica
of the command line used.  This is specifically useful for the -s option.
Since if -s is not specified, the user will not know what seed was used.
After all the tour finders are run, tsp_solve will always display the summary
of the performance of all tour finders run, that shows the Tour Finder name,
the number of tours found, and the total time used by the tour finder in
milliseconds and the cost of the tour found (or the average and standard
eviation of the tour costs found if more than one tour is found.)

Other options are available for more tour output detail.

@table @samp
@item -p1
Display the tour(s) found in gnuplot format.  The only output will be: (for each
tour) a list of x y coordinate numbers of the cities followed by the first x y
coordinate to complete the plot of the tour.  If a non-euclidean matrix is used
then only the city id's of the tour are output.  The output of the tour is
sent through stdout, all other output mentioned in all other options is sent
through stderr.

@item -p2
Display a line for each tour finder each time it is run.  Each tour's output
line will contain the tour finder's name, the cost of that tour and the time
it took to find that tour.

@item -p3
Display the same as -p2 but include a ordered output of the citie of the tour.

@item -p4
Display each tour found in TSPLIB format; the reverse of the `---' option.
@end table

@node Advanced Options
@section Advanced Options

The parallelizing options -n and -r will not be fully functional until
version 1.4.  However, the -r option when used alone will be able to
restore files created by sending a SIGUSR1 signal to the tsp_solve process
currently running the "imponetree" tour finder.

@table @samp
@item -nNFSDIR
When -n is specified, tsp_solve goes into client mode and uses the NFSDIR to
look for other clients currently running on the mounted directory.  This
option will enable a single TSP to be solved in parallel using multiple
machines that have a common mounted filesystem with directory NFSDIR.
Currently, "imponetree" is the only tour finder capable of runing in parralel
with other tsp_solve processes running.  To start a parallel algorithm,
first run tsp_solve with all the options needed to solve the large TSP on
one processor in addition to the -nNFSDIR option.  Then on any other machine
that has the same mount of the directory specified when starting the run,
invoke "tsp_solve -nNFSDIR" and it will attempt to split the currently
running queue and start running on the split queue.  The main client or the
tsp_solve client that was started first will report the best solution and
terminate the remaining clients after all clients return with no queues left
to explore.

The following files are created when running the "imponetree" algorithm
in parallel mode.  'X' in the [filename] will specify a numerical digit.
Each TSP started in parallel mode, is given a session number that may be
contained in the filenames.  [Xtour.tsp] (where X is the session number)
is the best tour found stored in a binary format.  [HHX-X.tsp] (where
two characters from the hostname are HH and the first X is the client id
and the second X is the Xth file made from this client so far) is a info
file passed with the -r option.  [tsphosts] is the host text file which
keeps track of what is running and how.  The format contains numbers and
character strings separated by spaces or tabs.  The first line is reserved
for the number of clients.  The remaining lines are for the clients, one
line per client.

The following information is what is on each line for each client: Session Id,
Client Id, Status bits, Process Id, USR1 Signal, USR2 Signal, tsp_solve
parameters with whitespace replaced by '^', Full Pathname, Hostname.  Here is
an example of a hosts file with one client running.

@example
1
0  0   128 24331 16 17 crash.cts.com -rcr0-1.tsp^100^imp tsp_solve
@end example

A single '^' character string specifies the null string.  In this example
the tsp_solve Full Pathname is blank. 

@item -rINFOFILE
The -r option is used to recover and continue a partial instance of a
previously running problem from an INFOFILE saved from a previous tsp_solve
process.

If the tsp_solve process catches a SIGUSR1 signal and certain algorithms
which have saving capability are running, the files state000.tsp and
tspinfo.txt are written to the current directory which preserves the full
state of the algorithm.  The -r option will continue the tour finder saved in
this state.  The restorable .tsp file created by a SIGUSR1 is usually a
megabyte, depending on the size of the core memory of the tsp_solve process
and how large a TSP is being run.

If the tsp_solve process catches a SIGUSR2 signal and certain algorithms
which have spliting capability are running and the tsp_solve process was
started with the -n option, a file is saved in the [HHX-X.tsp] format above.
This too can be restorable but it will only be a portion of "work" the previous
tsp_solve process has now ignored to do since it was asked to split and save
the partial "work" in the file.

@item -iN
Sets the initial choice of some algorithms to start with city N.  However, most
algorithms don't have a selectable starting city.  The default value is 0.

@item -oN
Sets the upper bound for branch and bound algorithms.  Usually the branch and
bound algorithm, will find it's own upper bound and use that, however if you
know ahead of time that there exists a tour that has a cost of C, that is a
better tour than a heuristic can give, then setting -o(C+1) will make the
branch and bound go a bit faster.  Note: if you call -o with a value that is
less than or equal to the cost of the best possible tour for the TSP, a branch
and bound algorithm will fail and not find a tour.  If -o is entered with out
the N argument, the upper bound is set to the cost of the in order tour.  If
the N argument is left out or is zero, the default upper bound is set to the
cost of the best heuristic tour.

@item -pN
This option determines what the printed tour output format of tsp_solve will
be.  See OUTPUT OPTIONS.  The default is 0.

@item -mN
This option determines what input format tsp_solve will expect.
See Input Options.  The default is 1.

@item -sN
Sets the initial seed value for generating random TSP matrices.  This option is
useful for invoking tsp_solve twice on the "same" randomly generated matrix.
The default seed is the return value of time(1) called before any algorithm is
run by tsp_solve.

@item -tN
Sets the number of times the algorithms are run on a new randomly generated
matrix.  If -t2 -s0 is passed on the command line, tsp_solve runs the tour
finders on a matrix generated with a seed of 0 and then runs the tour finders
on a new matrix generated with a seed of 1.  If -t is used with user input
TSPs, tsp_solve will expect data for N matrices to be passed through standard
input.  The default is 1.

@item -xN
Sets an Extra value, reserved for passing additional information to developing
tour finders.

@item -vN Sets the verbosity level.  Useful for debugging tour finders.  The
default is 0 and shows no extraneous information.

@item -dX
Override the default distance function for Euclidean TSPs.  The possible
values of "X" are "EUC_2D", "EST_2D", "GEO", "ATT", and "HOTO".  They each
represent a function that can be calculated from two coordinate sets that
result in a type of distance returned.  The default distance is "EST_2D"
which is a strict un-rounded pythagarean distance between two points.  The
other distance function are explained in the TSPLIB distribution.  If the
`---' option is used with this option, the X replaces the value of
'EDGE_WEIGHT_TYPE' in the .tsp file input.
@end table

@node Signal Responses
@section Signal Responses
Certain signals, when sent via kill(1), to tsp_solve, will interrupt or
spawn new things to happen with what tsp_solve is currently doing.  Most
signals are defined using signal(1) definitions.  Please refer to the signal
man page on your site to know the exact signal numerical values.

@table @samp
@item SIGUSR1
A SIGUSR1 signal asks tsp_solve to save the information it is doing
so it can be restored and continued when using the -rFILENAME option.  In
version 1.4 only the "imponetree" tour finder may be saved and restored in
this manner.

@item SIGUSR2
A SIGUSR2 signal asks the tsp_solve running to split its work because another
tsp_solve might be waiting to do something.  If the process finds that there
is another waiting client in the tsp hosts file, tsp_solve will save a
restorable file and put a "todo" entry in the hosts file for the waiting
client to pick it up.

@item SIGTSTP
If tsp_solve is interrupted with a SIGTSTP, care is taken to keep track of the
time lost while suspended between calculating how long a tour finder takes to
run.  If this signal is caught, it will print a tour summary of what tours were
found so far, and then stop the current tour finders timer enabling the process
to be truly suspended.  The SIGSTOP signal, since it cannot be caught in all
platforms, will not perform the previous function but mearly stop the process
leaving the tour finder timer running.

@item SIGCONT
If a SIGTSTP signal is caught and then a SIGCONT is caught while the tsp_solve
process is suspended, the process is restarted on running the current tour
finder (continuing the timer and the process.)  If the process was stopped by
SIGSTOP, then SIGCONT mearly resumes the process.
@end table


@node Tour Finders
@chapter Tour Finders

Tour finders are specified on the command line by name.  Unique abbreviated
prefix of the full names of tour finders may be used to specify tour finders
if there is no other tour finder with the same abbreviated prefix; "on" may
be used for "onetree" since there is no other tour finder that begins with
"on."

tsp_solve will execute each tour finder in the command line for each TSP
input.  The output format of the tour finder is determined by the output
options below.  Each tour finder will find and/or print it's solution after
it is executed.  Some tour finders are restricted to finding solutions for
only certain kinds of matrices.  The so far implemented tour finders are as
follows.

@example
TourFinder   |  optimal? | Types of TSPs that work best
-------------+-----------+-----------------------------------
best         |    yes    | any
imponetree   |    yes    | Symmetric
onetree      |    yes    | Symmetric
assign       |    yes    | Asymmetric
standard     |    yes    | any (slowest)
arborescence |    yes    | Almost Symmetric
hungarian    |    yes    | any small
heurbest     | heuristic | any
addition     | heuristic | any
farinsert    | heuristic | any
easytour     | undeterm. | any (in order)
random       | undeterm. | any (in random order)
flexmap      | heuristic | large Euclidean
disperse     | heuristic | Asymmetric, Non-Euclidean Symmetric
multifrag    | heuristic | any
antheur      | heuristic | any
splitadd     | heuristic | any
christofide  | heuristic | Symmetric
patching     | heuristic | Assymetric
psort        | heuristic | Assymetric
loss         | heuristic | Assymetric
@end example

"best" will choose the fastest currently implemented optimal solution tour
finder available, with respect to the type of TSP that has been input.
"assign" is an Assignment Problem Relaxation optimal tour finder.  It is
best for asymmetric TSPs.  "onetree" and "imponetree" are implementations
of the Volgenant and Jonker (1981) paper that uses the 1-Tree Langrangean
Relaxation.  "imponetree" does better for more random cities and uses
the enhanced branching techniques, whereas "onetree" does better for
pathelogical cases (grids, circles of cities) and does not use special
branching.  "arborescence" solves asymmetric TSPs best when the cost
from city i to city j is fairly close to the cost from city j to city i.
The "arborescence" algorithm uses a simliar Lagrangian training mechanism
to "imponetree".  It uses a relaxation algorithm modeled from the paper
"An Efficient Algorithm for the Min-Sum Arborescence Problem on Complete
Digraphs" by Matteo Fischetti and Paolo Toth, ORSA Journal on Computing,
1993, p426-434.  "hungarian" is the Hungarian method for optimally solving
TSPs and was implemented by Bob Craig.

"heurbest" tries to guess from the matrix which heuristic will find the best
possible heuristic solution with a minor reguard for speed.  "addition"
is a nearest addition tour finder.  If a Euclidean TSP is input, then
the addition heuristic will generate a convex hull as a initial tour.
"farinsert" is an implementation of the Farthest Insertion algorithm and
also generates a convex hull to start with if the problem is Euclidean.
"easytour" will return a tour as the order of cities input to tsp_solve.
"easytour" is a method of passing a user defined tour into tsp_solve so
other improvement algorithms may improve upon the tour input.  "flexmap"
does well for large Euclidean TSPs and is an implementation from the
algorithm and paper by Bernd Fritzke and Peter Wilke, 1991 IJCNN-91
Singapore.  "disperse" is an algorithm from an Academic Thesis "Traveling
Salesperson Dispersion: Performance and Description of a Heuristic" by
Chad Hurwitz, 1992.  "multifrag" is a heuristic based on the paper by Jon
Louis Bentley.  The "multifrag" algorithm is a watered down version from
the rigorus "mf" algorithm in the paper.  The "splitadd" heuristic is a
hybrid between the farthest insertion and nearest addition algorithms, for
the first 1/4 of the cities the algorithm chooses the farthest insertion,
for the rest the nearest addition algorithm is used to choose which city
to be inserted in the tour.  The "antheur" is heuristic that models
many ants trying to find a better path collectively.  This approach
is based on the paper: "The Ant System: An Autocatalytic Optimizing
Process" authored by Marco Dorigo, Vittorio Maniezzo, Albero Colorngo,
dorigo@@iprnel2.elet.polimi.it, maniezzo@@iprnel2.elet.polimi.it,
submitted to IEEE Transactions on Systems, Man, and Cybernetics.
"christof" (Christofide's algorithm), "patching" & "psort" (the Assigment
Problem's cycles patched together), and "savings" (a savings heuristic)
were taken from the excellent resource: The_Traveleing_Salesman_Problem, E.
Lawler, & etc., 1985.  The "loss" tour finder is an implementation of the
Loss method from the paper by P. Van Der Cruyssen and M. J. Rijckaert from
Journal of the Operational Reseach Society, 1978, Vol.29 No.7, p697-701.

@node Tour Improvements
@chapter Tour Improvements

There can also be tour improvements made to tours found by tour finders.
This can be specified on the command line by defining a tour finder as
"tour_finder+tour_improvement"; specifying a tour finder name immediately
followed by a '+' sign immediately followed by a tour improvement name.
Multiple improvements are allowed for as many '+TourImprover' are
attached to the tour finder name.  So far we have the following
tour improvement algorithms.

@example
TourImprover |  optimal? | Types of TSPs that work best
-------------+-----------+-----------------------------------
threeopt     | heuristic | any
fouropt      | heuristic | Asymmetric
k3swap       | heuristic | any
k5swap       | heuristic | Symmetric
k10swap      | heuristic | Any
k14swap      | heuristic | Asymmetric
k17swap      | heuristic | Symmetric
ropt         | heuristic | Symmetric
@end example

"threeopt" is the 3-Opt improvement scheme for Asymmetric or Symmetric TSPS.
"fouropt" is the 4-Opt for Asymmetric TSPs.  "k3swap" is an algorithm which
linearly goes down the tour looking at every adjacent k=3 cities and
re-arranges k cities within the tour until in a permutation that makes the
tour the best it can be.  "k14swap" or any other kXswap algorithm does the
same thing as k3swap but for k=X cities.  The Symmetric kswaping heuristics
are Slower than the Asymmetric ones. "ropt" is an implementation of the 1973
Lin and Kernighan paper, however without back tracking.

@node Messages
@chapter Messages

@table @samp
@item Could Not Run Tour Finder : XXX
Is triggered if the matrix input is not able to be run with the tour finder.
Usually, there is an explanation why the matrix can't be used with the given
tour finder.

@item No Tour Found
The tour finder could not find a tour given the matrix.  This is usually a bug
in the tour finder or an improper specification of the -o option with branch
and bound tour finders.

@item No Tour Finders Specified
tsp_solve cannot run with out specifying a tour finder on the command line.
see TOUR FINDERS.

@item Aborted at SEED ???, Sig#???.  So far the Timers show:
A segfault, arithmetic exception, bus error or other abortive signal has
occurred.  The Times, Tour lengths (for the algorithms run up to the
abortion), and the last seed value used to seed the random number generator
are printed.  If the program wasn't user terminated, it's time to start
debugging.
@end table

@node Examples
@chapter Examples

@table @asis
@item tsp_solve 10 addition
Finds a heuristic solution found from running the "addition" tour finder
using a randomly generated Euclidean TSP of 10 cities.

@item tsp_solve 10 farinsert -m3
Finds a heuristic solution found from running the "farinsert" tour finder
using a randomly generated Asymmetric TSP of 10 cities.

@item tsp_solve 80 di+th+k14 loss+th+k14 -m4 -t100
Generates 100 random Asymmetric oneway 80 city TSPs, and for each finds a
dispersion heuristic tour solution and a loss heurisitc tour solution, and
for both performs the threeopt improvement algorithm on each tour, and then
performs the k14swap improvement on the resulting tours.  The output is
the average and standard deviation of the sums of the tour lengths generated
by both tour finder.

@item tsp_solve 10 st far -m2
Finds an optimal and heuristic solution from the same randomly generated
Symmetric TSP of 10 cities.

@item tsp_solve 30 -dEUC_2D best heur -t10
Finds 10 optimal solutions and 10 heuristic solutions from the "heurbest"
heuristic from 10 randomly generated Euclidean TSPs of 15 cities using EUC_2D
type distances.

@item tsp_solve 100 b = < 100.coord
Finds the optimal solution to a 100 city TSP input from the file "100.coord" in
coordinate format.

@item tsp_solve 20 standard - < 20.matrix
Finds the optimal solution to a 20 city TSP input from the file "20.matrix" in
matrix format.

@item cat kroA100.* | tsp_solve --- e
Given TSPLIB input show the tour found using the "easytour" tour finder.
kroA100.tsp and kroA100.opt.tour are files from TSPLIB.

@item cat kroA100.* | tsp_solve --- heur -p4 | tsp_solve --- e
This exemplifies the -p4 output format can be read in by the `---' option.
The entire invokation will find a heuristic from the kroA100 TSP and then the
second piped invokation will interpret the TSPLIB output and run "easytour"
to extract the tour in the order it was inputed.

@item cat Coordinates | tsp_solve 100 e+th = -p1
Read in a tour defined by the file 'Coordinates' (interpreted by the "easytour"
tour finder), use the "threeopt" algorithm to improve upon that tour and then
print out the improved tour's coordinates.

@end table

@node Notes
@chapter Notes

As of yet there seems to be no published and public domain way to optimally
solve large non-trivial TSPs with out Linear Programming of more than
200 cities quickly.  If a program is needed find the optimal solution to
a large TSP, most likely tsp_solve will run for years if too large of a
TSP is specified.  However, tsp_solve can solve general medium size TSPs
giving guaranteed optimal solutions as well as provide decent heuristic
solutions for large TSPs.  To tell if the TSP is too large for this program
to solve optimally: have a go at it and see how far it gets by sending
a SIGUSR1 signal to the running process and view the tspinfo.txt file.

@node Bugs
@chapter Bugs

tsp_solve is geared to fail if it doesn't think it can do the job it was
asked.  The ANSI assert() macro was used un-conservatively through out
tsp_solve to verify the accuracy of the result.  If tsp_solve returns with
a solution and exit code 0, you can be very sure the solution is accurate.
However, tsp_solve may flunk an assertion and return a "programmer" useful
message.  To Date, tsp_solve has been run on hundreds of millions of randomly
generated TSPs with cities numbers ranging from 5 to 50 without assertion
failures.  In addition, many large TSPs have been solved from the TSPLIB data
set library.  But, if an assertion does fail, please mail the TSP input,
command line options, and assertion message to churritz@@cts.com.

@node Contributors
@chapter Contributors

The following identities are of code contributors to this project.

@example
churritz@@cts.com
(619) 565-8656
Chad Hurwitz
@end example

@example
kat3@@ihgp.ih.lucent.com
AT&T Bell Labs
R. J. Craig
Provided: Hungarian Algorithm
@end example

The following identities deserve Special Thanks for help provided:

@example
fritzke@@immd2.informatik.uni-erlangen.de
Universit:at Erlangen-N:urnberg
Bernd Fritzke and Peter Wilke
@end example

@example
dkraay@@hera.mgmt.purdue.edu
David Kraay
@end example

@bye
